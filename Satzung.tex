

\begin{abstract}
	Seit mehreren Generationen existiert in Schöffengrund-Oberquembach eine Burschenschaft, die sich bisher aktiv und regelmäßig am Dorfleben beteiligte. Diese Burschenschaft richtet seit Generationen jedes Jahr immer zum Erntedankfest die zur Tradition gewordene Kirmes aus. Des Weiteren wird seit Generationen vor und im historischen Backhaus und vor der Kirche am Dorfweiher das Aufstellen des Maibaums mit einem kleinen Fest begangen. Nach Aufzeichnungen existierte die Burschenschaft seit Anfang des zwanzigsten Jahrhunderts. Ein genauer Gründungstermin konnte bisher nicht festgestellt werden. Nach den Aufzeichnungen haben die Kirmesburschen sogar während des ersten und zweiten Weltkriegs anlässlich des Erntedankfestes Kirmesaktivitäten an der Dorflinde und in den Gasthäusern organisiert. Im Jahre 1975 gab sich die Burschenschaft den Namen Schoppenelf und es konnten auch gleichzeitig Mädchen der Burschenschaft beitreten.\\
	\\
	Strukturelle und kulturelle Lage in Oberquembach
	Durch den strukturellen und  demographischen Wandel ist bei allen Oberquembacher Ortsvereinen (MGV Liederkranz 1886 Oberquembach, Frauenchor Oberquembach, Freiwillige Feuerwehr Oberquembach und der SG Oberquembach) ein deutlicher Nachwuchsmangel zu verzeichnen, sollte dem nicht entgegengewirkt werden ist zu befürchten, dass in den nächsten Jahren die Vereine ihre originäre Tätigkeit nicht mehr ausführen können. Auch bei der Burschenschaft ist zu befürchten, dass in naher Zukunft immer weniger Jugendliche der Burschenschaft beitreten. Um die Burschenschaft neu auszurichten, das Vereinsleben in Oberquembach neu zu beleben und zusätzliche Mitglieder zu gewinnen, hat die am 28.02.2015 im Dorfgemeinschaftshaus  Oberquembach stattgefundene Mitgliederversammlung folgende Satzung beschlossen.  
\end{abstract}

\pagebreak

%§1
\section{Name, Sitz, Zweck des Vereins} \label{1}
Der Burschenschaft „Schoppenelf“ Oberquembach (e.V.)
mit Sitz in Oberquembach verfolgt ausschließlich und unmittelbar gemeinnützige Zwecke im Sinne des Abschnitts „Steuerbegünstigte Zwecke“ der Abgabenordnung.
Zweck des Vereins ist Förderung von Jugendhilfe, Erziehung, Kunst und Kultur, Landschaftspflege, Umweltschutz des öffentlichen Gesundheitswesens und des Sports.
Der Satzungszweck wird verwirklicht insbesondere durch Pflege des Lied- und Kulturgutes, Förderung sportlicher Leistungen, Pflege und Erhaltung der Umwelt, Jugendförderung, Unterstützung der Kirchengemeinde.

%§2
\section{}
Der Verein ist selbstlos tätig; er verfolgt nicht in erster Linie eigenwirtschaftliche Zwecke.

%§3
\section{}
Mittel des Vereins dürfen nur die satzungsmäßigen Zwecke verwendet werden. Die Mitglieder erhalten keine Zuwendungen aus Mitteln des Vereins.

%§4
\section{}
Es darf keine Person durch Angaben, die dem Zweck der Körperschaft fremd sind, oder durch unverhältnismäßig hohe Vergütung begünstigt werden.

%§5
\section{Erntedankkirmes}	
Jeweils am 1. Oktoberwochenende hält die Burschenschaft die traditionelle Kirmes ab. Hierbei wird die Kirchengemeinde eingeladen, einen Erntedankgottesdienst durchzuführen. Des Weiteren veranstaltet die Burschenschaft im Rahmen der Kirmes einen Dorfgemeinschaftsabend, in der Oberquembacher platt gesprochen werden soll. Die beiden Oberquembacher Chöre (MGV Liederkranz 1886 Oberquembach, Frauenchor Oberquembach) werden gebeten mit Liedbeiträgen an diesem teilzunehmen. Zu besonderen Jubiläen wird ein Kirmesumzug, an dem alle Oberquembacher Vereine teilnehmen, durchgeführt.

%§6
\section{Traditionelles Maibaumstellen}
Jeweils zum 1. Mai richtet die Burschenschaft das traditionelle Maibaumstellen aus. Hierzu werden die Räume des historischen Backhauses und der Back Raum mit Backofen genutzt. Die beiden Oberquembacher Chöre (MGV Liederkranz 1886 Oberquembach, Frauenchor Oberquembach) sind einzuladen, um mit Liedbeiträgen das traditionelle Maifest zu begleiten.

%§7
\section{Traditionelle Fastnachtssitzung}
Am Fastnachtswochende  führt die Burschenschaft im Dortfgemeinschaftshaus die traditionelle Fastnachtssitzung durch. Ziel ist es, dass mindestens  die Hälfte der Vorträge bzw. Reden in der Mundart (Oberquembacher Platt) abgehalten werden.

%§8
\section{Fußballturnier}
Einmal im Jahr richtet die Burschenschaft ein Fußballturnier auf dem Oberquembacher Sportplatz aus.
Ziel ist es hierbei die Oberquembacher Jungen und Mädchen für den Fußball zu begeistern und für die Fußballmannschaften SG Oberquembach zu gewinnen.

%§9
\section{Dreschplatz Oberquembach}
Patenschaft zur Pflege und Erhaltung des Dreschplatzes in Oberquembach.

%§10
\section{Ferienpassaktion}
Jährliche Teilnahme an der Ferienpassaktion der Gemeinde Schöffengrund. Dies beinhaltet eine Aktion mit Kindern und Jugendlichen in der Zeit der Sommerferien.

%§11
\section{Geschäftsjahr}
Geschäftsjahr ist das Kalenderjahr.

%§12
\section{Eintragungsabsicht}
Die Burschenschaft „Schoppenelf“ Oberquembach beabsichtigt eine Eintragung in das Vereinsregister.

%§13
\section{Erwerb und Beendigung der Mitgliedschaft}
	\subsection{} Voraussetzung für den Erwerb der Mitgliedschaft ist ein, an den Vorstand des Vereins zu richtender, Aufnahmeantrag, in dem sich der Antragssteller zu Einhaltung der Satzungsbestimmungen verpflichtet. Der Vorstand entscheidet über die Aufnahme nach freiem Ermessen.
	
	\subsection{} Die Mitgliedschaft endet mit dem Tod, schriftlicher Austrittserklärung und Ausschließung. Ein Mitglied kann jederzeit seinen Austritt durch schriftliche Erklärung an den Vorstand erklären. Die Mitgliedschaft endet somit zum Ende des laufenden Geschäftsjahres. 
	Die Ausschließung ist zulässig, wenn das Mitglied schuldhaft in grober Weise die Interessen des Vereins verletzt. Über den Ausschluss entscheidet die Mitgliederversammlung. Dem Mitglied ist vor der Mitgliederversammlung eine Möglichkeit zur Stellungnahme zu den Vorwürfen zu geben. Dazu ist eine Frist, die zwei Wochen vor der Mitgliederversammlung endet, zu setzen. Die Ausschließung eines Mitgliedes ist ebenfalls zulässig, wenn es seiner Pflicht zur Zahlung des Mitgliedsbeitrages nicht nachkommt. Näheres hierzu regelt \ref{14}.
	
	\subsection{} Mitglieder erhalten keine Gewinnanteile und in ihrer Eigenschaft als Mitglieder auch keine sonstigen Zuwendungen aus den Mitteln des Vereins. Ausgenommen sind Geschenke zu Hochzeiten, Jubiläen oder ähnlichen Anlässen. Ein Mitglied hat nach Beendigung der Mitgliedschaft keinen Anspruch auf das Vereinsvermögen. 
	
	\subsection{} Alle Mitglieder der Burschenschaft „Schoppenelf“ Oberquembach werden mit der Eintragung in das Vereinsregister in die neu gegründete Burschenschaft „Schoppenelf“ Oberquembach e.V. übernommen.

%§14
\section{Mitgliedsbeiträge}	\label{14}
Die Mitgliedschaft verpflichtet zur Förderung des Vereinszweckes und zur Zahlung der Mitgliedsbeiträge. Der Mitgliedsbeitrag wird von der Mitgliederversammlung festgesetzt.
Er ist im laufenden Geschäftsjahr zu entrichten. Kommt ein Mitglied seiner Pflicht zur Beitragszahlung nicht nach, kann es im Ermessen des Vorstandes vom Vorstand mit sofortiger Wirkung aus dem Verein ausgeschlossen werden. Das somit ausgeschlossene Mitglied ist schriftlich zu benachrichtigen. Es hat die Möglichkeit an der nächsten ordentlichen Mitgliederversammlung Widerspruch gegen den Ausschluss einzulegen. Somit sind zwei Drittel der Stimmen der Anwesenden Mitglieder nötig, um das Mitglied auszuschließen. Bis zur Entscheidung der Mitgliederversammlung ruht die Mitgliedschaft des betroffenen Mitglieds. 

%§15
\section{Vorstand}
	\subsection {} Die Geschäfte des Vereins werden vom Vorstand geführt, der sich wie folgt zusammensetzt:
	\begin{itemize}
		\item Vorsitzender und Stellvertreter 
		\item Kassenwart und Stellvertreter
		\item Schriftführer und Stellvertreter 
		\item Sport- und Wagenwart
		\item Erweiterter Vorstand bestehend aus drei bis fünf Beisitzern und zwei Beitragskassierern  
	\end{itemize} 
	
	\subsection {}  Die Mitglieder des Vorstandes werden jeweils in der ordentlichen Mitgliederversammlung für die Dauer von zwei Jahren gewählt durch Handaufheben oder geheime Wahl. Der Vorstand bleibt jedoch so lange im Amt, bis ein neuer Vorstand gewählt ist. Die Wiederwahl eines Vorstandsmitgliedes ist zulässig. 
	
	\subsection {} Der Verein kann gerichtlich und außergerichtlich vom Vorstandsvorsitzenden , seinem Stellvertreter, Schriftführer und Kassenwart vertreten werden. Der Vorstand behält es sich vor, für bestimmte Rechtsgeschäfte, einzelnen Mitgliedern Vertretungsvollmachten zu erteilen. Diese sind in schriftlicher Form zu erteilen und vom Vorstandsvorsitzenden und seinem Stellvertreter zu unterschreiben. Näheres hierzu regelt die Geschäftsordnung. 
	Entscheidungen innerhalb des Vorstandes müssen mit einfacher Mehrheit gefällt werden und das Ergebnis ist schriftlich in einem Protokoll festzuhalten. Der Vorstand ist bei der Anwesenheit von mindestens sechs Vorstandsmitgliedern beschlussfähig. 
	
	\subsection {} Scheidet ein Vorstandmitglied, jedoch nicht der Vorstandsvorsitzende, während einer Amtsperiode aus, so kann vom verbleibenden Vorstand, falls dieser noch aus mindestens sieben Mitgliedern besteht, ein Ersatzmitglied gewählt werden. Andernfalls entscheidet die Mitgliederversammlung. Tritt der Vorstandsvorsitzende ab, wählt die Mitgliederversammlung einen Nachfolger. 

%§16
\section{Mitgliederversammlung}
	\subsection{} Die ordentliche Mitgliederversammlung des Vereins findet jeweils in den ersten drei Monaten des folgenden Geschäftsjahres statt. Die Mitgliederversammlung beschließt insbesondere über
	
	\begin{itemize}
		\item die Festsetzung der Mitgliedsbeiträge 
		\item die Wahl und Abberufung der Vorstandsmitglieder
		\item den Ausschluss eines Mitglieds
		\item die Auflösung des Vereins
		\item die Wahl von zwei Kassenprüfern (alljährlich, keine direkte Wiederwahl möglich)
		\item Änderung der Satzung  
	\end{itemize} 

	
	\subsection{} Die Mitgliederversammlung wird durch einen Aushang im Schöffengrunder Blättchen einberufen. Dieser Aushang beinhaltet auch die Tagesordnung der Mitgliederversammlung.  
	
	\subsection{} Bei Beschlussfassung in den Mitgliederversammlungen entscheidet die Mehrheit der erschienen Mitglieder. Zur Ausschließung eines Mitgliedes und zur Änderung der Satzung, außer \ref{1} und \ref{17}, sind zwei Drittel der Stimmen und zur Änderung von \ref{1} oder \ref{17} der Satzung Einstimmigkeit der erschienenen Mitglieder notwendig. Die Abstimmung zur Auflösung des Vereins regelt \ref{17}. Gefasste Beschlüsse sind von der Schriftführerin in einem Protokoll festzuhalten und vom Vorstandsvorsitzenden sowie seinem Vertreter zu unterzeichnen.  
	
	\subsection{} Außerordentliche Mitgliederversammlungen können vom Vorstand einberufen werden, oder wenn mindestens 10\% Mitglieder dies wünschen. 
	
	\subsection{} Anträge zur Tagesordnung der ordentlichen Mitgliederversammlung können  
	bis zum Ende des jeweiligen Geschäftsjahres gestellt werden. Zusätzlich sind in der Mitgliederversammlung Themen die ein Mitglied besprechen will oder Fragen die ein Mitglied stellt, die nicht den aktuellen Tagesordnungspunkt betreffen, unter dem Tagesordnungspunkt „Verschiedenes“ zu behandeln.

%§17
\section{Auflösung des Vereins}	\label{17}
Die Auflösung des Vereins bedarf des Beschlusses der Mitgliederversammlung mit einer Mehrheit von drei Vierteln der erschienenen Mitglieder.
Die Auflösung des Vereins führt der jeweils amtierende Vorstand durch. Das Vereinsvermögen, das nach Begleichung aller ausstehen Verbindlichkeiten übrig bleibt, kommt mindestens zur Hälfte dem Förderverein Pflegestation Schöffengrund e.V. zu Gute. Das restliche Vermögen kann vor der Liquidation noch zu satzungsgemäßen Zwecken genutzt werden. 
















